\documentclass[titlepage]{ltjsreport}
\usepackage{luatexja, graphicx, multirow, listings, url, hyperref, comment}
\makeatletter
\renewcommand{\abstractname}{アブストラクト}
\newcommand{\image}[2]{\begin{figure}[h]\vspace{1em}\begin{center}\includegraphics[width = .9\textwidth]{#2}\caption{#1}\label{#1}\end{center}\end{figure}}
\newenvironment{mytable}[2]{\begin{table}[h]\begin{center}\caption{#1}\label{#1}\begin{tabular}{#2}}{\end{tabular}\end{center}\end{table}}
\newcommand{\inputtextbox}[3][]{\vspace{1em}\lstset{language = #1}\lstinputlisting[caption = #2, label = #2]{#3}}
\lstnewenvironment{textbox}[2][]{\vspace{1em}\lstset{language = #1, label = #2, caption = #2}}{}
\renewcommand{\lstlistingname}{\figurename}
\makeatother
\lstset{language = , commentstyle = {\itshape}, stringstyle = {\ttfamily}, frame = {tblr}, breaklines = true, numbers = left, xleftmargin = 5.5em, framexleftmargin = 2.5em, numberstyle = {\scriptsize}, captionpos = b}
\begin{document}
\makeatletter
\let\c@lstlisting=\c@figure
\makeatother
\begin{titlepage}
\vspace*{1em}
\begin{center}
\begin{huge}
情報工学実験
\\
\vspace{1em}
に関する実験
\end{huge}
\end{center}
\vspace{6em}
\begin{flushright}
\begin{LARGE}
情報工学科 年 番
\\
\vspace{1em}
名前
\end{LARGE}
\end{flushright}
\vspace{4em}
\begin{flushleft}
\begin{Large}
\begin{tabular}{ll}
提出期限: & //():
\\
提出日: & //()
\\
\end{tabular}
\\
\vspace{2em}
共同実験者:班
\\
\vspace{1em}
\begin{tabular}{ll}
出席番号 & 名前
\\
出席番号 & 名前
\\
\end{tabular}
\end{Large}
\end{flushleft}
\end{titlepage}
\begin{abstract}
\end{abstract}
\chapter{解決方法}
環境は\cite{環境}のEnvironmentクラスに実装されている。

環境は12人が4部屋へどのように割り当てられているかを保持しているが,エージェントはそれを直接観測する訳ではない。

エージェントが観測する状態や環境が受け取る行動を定義するために,「attended rooms」「movable people」という概念を導入する。「attended rooms」とは環境が4部屋のうち2部屋をランダムに選択したものである。「movable people」とはattended roomsから1人ずつ,計2人ランダムに選択したものである。attended roomsには順序の概念があり,それぞれattended room 1・attended room 2とする。movable peopleも同様に,attended room 1から選ばれたmovable person 1とattended room 2から選ばれたmovable person 2の2人からなる。attended rooms・movable peopleを抽選する処理はEnvironmentクラスの\#drawメソッドに実装されている。\#drawメソッドを図\ref{attended rooms・movable peopleの抽選}に示す。この中で使われているsnatchRandomElement関数は引数のSetからランダムに1つ要素を取り出し,その要素を元のSetから削除する関数である。attended rooms・movable peopleの抽選はエージェントが環境に対して行動する度に行われる。
\begin{textbox}{attended rooms・movable peopleの抽選}
#draw() {
	const snatchedRooms = new Set([...this.rooms]);
	this.#attendedRooms = [];
	for (let i = 0; i < this.#a; i++) {
		this.#attendedRooms.push(snatchRandomElement(snatchedRooms));
	}
	this.#movablePeople = [];
	for (const attendedRoom of this.#attendedRooms) {
		const snatchedMembers = new Set([...attendedRoom.members]);
		for (let i = 0; i < this.#m; i++) {
			this.#movablePeople.push(snatchRandomElement(snatchedMembers));
		}
	}
}
\end{textbox}

エージェントが環境に対して状態を要求したとき,環境はmovable personに対して「同室希望」とした人の人数,「同室拒否」とした人の人数をattended room単位で集計する。例えば,movable person 1に対して「同室希望」とした人がattended room 2に2人,「同室拒否」とした人がattended room 2に1人いる場合,attended room 2からmovable person 1に対する集計結果は「同室希望:2 同室拒否:1」となる。環境がエージェントに対して返す状態は以下の4つを2×2の2次元配列に並べたものとなる。
\begin{enumerate}
\item attended room 1からmovable person 1に対する集計結果
\item attended room 2からmovable person 1に対する集計結果
\item attended room 1からmovable person 2に対する集計結果
\item attended room 2からmovable person 2に対する集計結果
\end{enumerate}

環境がエージェントに状態を提供する処理はEnvironmentクラスのgetStateメソッドに実装されている。getStateメソッドを図\ref{状態の提供}に示す。この中で使われているImpressionStatクラスはattended roomからmovable personに対する集計結果を表し,「同室希望」とした人数,「同室拒否」とした人数をそれぞれnumber型でフィールドに持つ。
\begin{textbox}{状態の提供}
getState(): ImpressionStat[][] {
	return this.#movablePeople.map((movablePerson) => {
		return this.#attendedRooms.map(function (attendedRoom) {
			return new ImpressionStat(
			[...attendedRoom.members].filter(function (member) {
				return member.likedPeople.has(movablePerson);
			}).length,
			[...attendedRoom.members].filter(function (member) {
				return member.dislikedPeople.has(movablePerson);
			}).length,
			);
		});
	});
}
\end{textbox}

エージェントが環境に対して行える行動はmovable peopleどうしの入れ替え,あるいは「何もしない」である。\cite{環境}には行動を表すActionクラスが定義されていて,1または2を代入可能なnumber型フィールドを2つ持つ。1・2あるいは2・1を持つAction型インスタンスはmovable person 1とmovable person2の入れ替えを表し,1・1,あるいは2・2を持つインスタンスは「何もしない」を表す。「何もしない」という行動を行った場合も,先述したattended rooms・movable peopleの抽選は行われる。

エージェントが環境に対して行動を行ったとき,幸福度の総和が変化するときがある。行動に対して環境がエージェントに返す報酬は,行動後の幸福度総和から行動前の幸福度総和を引いたものである。

\begin{thebibliography}{99}
\bibitem{環境} takechan-NITNC, "room-allocation-reinforcement-learning-environment", \url{https://github.com/takechan-NITNC/room-allocation-reinforcement-learning-environment}, 2023/12/12参照.
\end{thebibliography}
\end{document}